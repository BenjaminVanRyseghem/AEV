\documentclass[a4paper,10pt]{article}
\input{/Users/benjamin/Documents/Education/LaTeX/macro.tex}

\title{AEV: S�ance 7}
\author{Benjamin \bsc{Van Ryseghem}}

\begin{document}
\maketitle

\section{Exercice 1}
\subsection{Question 1}

\begin{itemize}
\item taille de la m�moire: $2^{16}$
\item cache de 32 lignes : $2^5$
\item nombre de mots par bloc : $2^3$
\end{itemize}

\paragraph{R�sultat:}
\begin{tabular}{|c|c|c|}
\hline
tag & ligne & mot\\
\hline
8 (\emph{16-ligne-mot}) & 5 & 3\\
\hline
\end{tabular}

\subsection{Question 2}
\paragraph{0001000100011011 :}
\begin{tabular}{|c|c|c|}
\hline
tag & ligne & mot\\
\hline
00010001 & 00011 & 011\\
\hline
\end{tabular}

Donc, c'est la 3\up{�me} ligne.

\paragraph{1100001100110100 :}
\begin{tabular}{|c|c|c|}
\hline
tag & ligne & mot\\
\hline
11000011 & 00110 & 100\\
\hline
\end{tabular}

Donc, c'est la 6\up{�me} ligne.

\paragraph{1101000000011101 :}
\begin{tabular}{|c|c|c|}
\hline
tag & ligne & mot\\
\hline
11010000 & 00011 & 101\\
\hline
\end{tabular}

Donc, c'est la 3\up{�me} ligne.

\paragraph{1010101010101010 :}
\begin{tabular}{|c|c|c|}
\hline
tag & ligne & mot\\
\hline
10101010 & 10101 & 010\\
\hline
\end{tabular}

Donc, c'est la 21\up{�me} ligne.

\subsection{Question 3}

L'octet d'adresse \emph{0001101000011010} est rang� dans la ligne 3.
Les adresses 1 et 3 de la question pr�c�dente sont rang�es au m�me endroit.

\subsection{Question 4}
Son tag, pour pouvoir retrouver la donn�e, et �tre sur qu'elle n'a pas �t� �cras�e.

\subsection{Question 5}
\fbox{
	\begin{minipage}{1\textwidth}
		\[\mbox{taille effective} = (\mbox{taille d'une ligne} + \mbox{taille d'un tag}) \times \mbox{nombre de lignes}\]
	\end{minipage}
}

\paragraph{Calcul de la taille effective : }
\begin{eqnarray*}
\mbox{taille effective} &=& (2^3 + 1) * 2^5\\
\mbox{taille effective} &=& 2^8+ 2^5\\
\mbox{taille effective} &=& 288 \ \mbox{octets}
\end{eqnarray*}

La taille effective est donc de 288 octets.

\section{Exercice 2}
\subsection{Question 1}
\begin{itemize}
\item taille de la m�moire: $64K = 2^{16}$
\item taille du cache : $1024: 2^{10}$
\item nombre de mots par bloc : $128: 2^7$
\item nombre de lignes : $\dfrac{\mbox{taille du cache}}{\mbox{nombre de mots par ligne}}=\dfrac{2^{10}}{2^7}=2^3$
\end{itemize}

\paragraph{R�sultat:}
\begin{tabular}{|c|c|c|}
\hline
tag & ligne & mot\\
\hline
6 (\emph{16-ligne-mot}) & 3 & 7\\
\hline
\end{tabular}

\subsection{Question 2}
\begin{itemize}
\item nombre total d'acc�s m�moire: $25836$
\item nombre de d�faut de cache: $10+4\times9+2 = 48$
\item nombre d'acc�s au cache: $25836 - 48 = 25788$
\end{itemize}
\paragraph{Temps d'acc�s moyen: }\[ \dfrac{48\mbox{M}+25788\mbox{m}}{25836}\]

\section{Exercice 3}
\subsection{Question 1}
\begin{itemize}
\item taille de la m�moire: $2^{32}$
\item nombre de lignes : $2^{14}$
\item nombre de mots par bloc : $1=2^0$
\end{itemize}

\paragraph{R�sultat:}
\begin{tabular}{|c|c|c|}
\hline
tag & ligne & mot\\
\hline
18 (\emph{32-ligne-mot}) & 14 & 0\\
\hline
\end{tabular}

\paragraph{Taille totale : } $(2^5+18)\times2^{14}$ bits

\subsection{Question 2}
\begin{itemize}
\item taille de la m�moire: $2^{32}$
\item nombre de lignes : $2^{(14-4)}=2^{10}$
\item nombre de mots par bloc : $16=2^4$
\end{itemize}

\paragraph{R�sultat:}
\begin{tabular}{|c|c|c|}
\hline
tag & ligne & mot\\
\hline
18 (\emph{32-ligne-mot}) & 10 & 4\\
\hline
\end{tabular}

\paragraph{Taille totale : } $(16\times2^5+18)\times2^{10}$ bits

\subsection{Question 3}
\begin{itemize}
\item taille de la m�moire: $2^{32}$
\item nombre de mots par bloc : $1=2^0$
\end{itemize}

\paragraph{R�sultat:}
\begin{tabular}{|c|c|}
\hline
tag & mot\\
\hline
32 (\emph{32-mot}) & 0\\
\hline
\end{tabular}

\paragraph{Taille totale : } $(16\times2^5+18)\times2^{10}$ bits



\subsection{Question 4}
\begin{itemize}
\item taille de la m�moire: $2^{32}$
\item nombre de blocs : $2^{(14-4)}=2^{10}$
\item nombre de mots par bloc : $1=2^0$
\end{itemize}

\paragraph{R�sultat:}
\begin{tabular}{|c|c|c|}
\hline
tag & ligne & mot\\
\hline
22 (\emph{32-ligne-mot}) & 10 & 0\\
\hline
\end{tabular}

\paragraph{Taille totale : } $(2^5+32)\times2^{14}$ bits

\subsection{Question 5}
\begin{itemize}
\item taille de la m�moire: $2^{32}$
\item nombre d'ensemble : $2^{(14-4-3)}=2^{7}$
\item nombre de mots par bloc : $8=2^3$
\end{itemize}

\paragraph{R�sultat:}
\begin{tabular}{|c|c|c|}
\hline
tag & ligne & mot\\
\hline
22 (\emph{32-ligne-mot}) & 7 & 3\\
\hline
\end{tabular}

\paragraph{Taille totale : } $(8\times2^5+22)\times2^{14}$ bits

\section{Exercice 4}
\subsection{Question 1}
\subsubsection{a)}

\begin{tabular}{|c|c|c|c|c|c|c|c|c|c|c|c|}
\hline
trace		&a	&b	&a	&c	&a	&b	&d	&b	&a	&c	&d	\\
default	&*	&*	& 	&*	&*	&*	&*	&	&*	&*	&*	\\
cache	&a	&a	&a	&c	&c	&b	&b	&b	&a	&a	&d	\\
		&\#	&b	&b	&b	&a	&a	&d	&d	&d	&c	&c	\\
\hline


\end{tabular}

\signature

\end{document}